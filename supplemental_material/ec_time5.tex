\section*{Figure \arabic{counter}}
\hypertarget{sec:ec_time5}{}

%%%%%%%%%%%%%%%%%%%%%%%%%%%%%%%%%%%%%%%%%%%%%%%%%%%%%%%%%%%%%%%%%%%%%%%%%%%%%%%%
% Output file: ec_time5.
%%%%%%%%%%%%%%%%%%%%%%%%%%%%%%%%%%%%%%%%%%%%%%%%%%%%%%%%%%%%%%%%%%%%%%%%%%%%%%%%

\noindent To make Figure \arabic{counter},
first run the nucleosynthesis \hyperlink{sec:single_zone}{calculation}.
Next, compile the NucNet Tools analysis codes by typing:
\begin{verbatim}
cd ~/projects/nucnet-tools-code/examples/analysis

make all_analysis
\end{verbatim}
Once that is done, return to the nuclear\_mass project:
\begin{verbatim}
cd ~/projects/nuclear_mass
\end{verbatim}
and run the code to compute nuclear mass terms during the calculation by
typing (all on one line):
\begin{verbatim}
./compute_mass_terms_in_zones ../nucnet-tools-code/data_pub/coulomb/my_output5.xml ../nucnet-tools-code/data_pub/coulomb/output_terms5.xml
\end{verbatim}
Now extract the $E_C$ per nucleon versus time by typing (all on one line):
\begin{verbatim}
../nucnet-tools-code/examples/analysis/print_properties ../nucnet-tools-code/data_pub/coulomb/output_terms5.xml time "Coulomb term" > ../nucnet-tools-code/data_pub/coulomb/ec_time5.txt
\end{verbatim}
Graph column 3 versus column 2 in {\em ec\_time5.txt}.

\addtocounter{counter}{1}

