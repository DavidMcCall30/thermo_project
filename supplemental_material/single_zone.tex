\section*{Nucleosynthesis Calculations} \hypertarget{sec:single_zone}{}

\hspace{4ex}To make Figure \arabic{counter},
you first need to run the nucleosynthesis calculation.
First follow the instructions at
\url{http://sourceforge.net/p/nucnet-projects/wiki/nuclear_mass/} to install
and compile the relevant codes.\footnote{Throughout these instructions, we will
assume you have installed the {\em projects/} directory off your home
directory.}

Once the {\em nuclear\_mass} project is properly installed and compiled,
you are ready to run the the nucleosynthesis calculations.  To do so,
change into the nucnet-tools-code directory that downloaded automatically
when you compiled the {\em nuclear\_mass} project and compile the single zone
network code:
\begin{verbatim}
cd ~/projects/nucnet-tools-code/examples/network

make all_network
\end{verbatim}
Next, download the data for the calculation by typing:
\begin{verbatim}
make data
\end{verbatim}
This creates a directory {\em ../../data\_pub}.  For convenience, create
a subdirectory under {\em ../../data\_pub} for your output called
{\em coulomb} by typing
\begin{verbatim}
cd ../../data_pub

mkdir coulomb
\end{verbatim}
Now edit
the file {\em zone.xml} in {\em data\_pub} until it reads

\begin{verbatim}
  <zone_data>
    <zone>
      <optional_properties>
        <property name="tend">1.e10</property>
        <property name="tau">0.2</property>
        <property name="munuekT">-inf</property>
        <property name="t9_0">5.</property>
        <property name="steps">5</property>
        <property name="rho_0">1.e7</property>
      </optional_properties>
      <mass_fractions>
        <nuclide name="si28">
          <z>14</z>
          <a>28</a>
          <x>1</x>
        </nuclide>
      </mass_fractions>
    </zone>
  </zone_data>
\end{verbatim}

Once this is done, you are ready to run the calculation.  Return to
{\em $\sim$/projects/nucnet-tools-code/examples/network}
by typing
\begin{verbatim}
cd ../examples/network
\end{verbatim}
and type (all on one line)
\begin{verbatim}
./run_single_zone ../../data_pub/my_net.xml ../../data_pub/zone.xml ../../data_pub/coulomb/my_output5.xml "[z <= 35]"
\end{verbatim}
This calculation uses the network data in {\em ../../data\_pub/my\_net.xml}
and the run data in {\em ../../data\_pub/zone.xml} and places the output
in {\em ../../data\_pub/coulomb/my\_output5.xml}.  The
\begin{verbatim}"[z <= 35]"\end{verbatim} selects a network that only includes
species up to $Z = 35$, which is certainly adequate for our purposes.
(Typical execution time for an x86\_64 type processor: $\sim$50 minutes.)

