\section*{Introduction}
The p-process involves proton-rich nuclides that are made by adding one or more protons to an atomic nucleus. This type of nuclear reaction is called the proton capture reaction. (aka p$/gamma$ reaction) When one adds a proton to a nucleus, the ratio of the protons and neutrons also change which results in a proton-richer isotope of the next element. 

This project will be about the  $92$-Molybdenum decay in the p process of a thermonuclear supernova.  This decay occurs after there is a shockwave generated from a supernova explosion. This shockwave goes through and causes heavier material in the leftover star to breakdown. Then the system cools and freezes out before all of the material breaks down.  Here we are looking for a possible shell to get the right temperature. If yhou look at the p process with core collapse and compare the amounts of molecules with respect to the oxygen. (The p process is too low for the solar system so we think the thermonuclear supernovas may be the answer to why we have the amount of oxygen in our solar system. First we will run a simple calcluation. Thjen we will look at doing multiple zones and grid the zones. If there is time we will turn to a core collapse (this uses the p process at two different sites.))

There is a paper that Dr. Meyer wrote as a post doc that you may want to check out. The paper is called {\bf A new site for the astrophysical gamma-process} by W. Howard and B. Meyer. 


%This supplementary material explains in some detail how to create the figures 
%in the parent paper.  We describe input files you need to edit and commands 
%that you type to make the data for the figures.

%\hspace{4ex}In what follows, you will need to type certain lines directly into 
%your computer.  For example,
%\begin{verbatim}
%./command input1 input2
%\end{verbatim}
%means that you should enter {\bf ./command input1 input2} into your computer.  
%We in fact recommend that you simply copy and paste such text (for example, try 
%triple-clicking the line to select it) into the appropriate directory.
